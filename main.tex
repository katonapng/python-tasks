\documentclass[a4paper]{article}
\usepackage[12pt]{extsizes} % для того чтобы задать нестандартный 12-ый размер шрифта
\usepackage[utf8]{inputenc}
\usepackage[russian]{babel}
\usepackage{setspace,amsmath}
\usepackage[argument]{graphicx}
\usepackage{hyperref}
\DeclareGraphicsExtensions{.pdf,.png,.jpg}
\usepackage[left=25mm, top=20mm, right=25mm, bottom=20mm, nohead, footskip=10mm]{geometry} % настройки полей документа
 
\begin{document} % начало документа
 \textbf{Задача 1}\\
 
 \textbf{Постановка}
 
 Назовём строку хорошей, если в ней содержаться только буквы 'd', 'g', 'o' и предыдущая буква не совпадает со следующей.\\
 
 Примеры хороших строк: 'dog', 'god', 'dogdog'\\
 
 Примеры плохих строк: 'good', 'ooo', 'doc'\\
 
 Даны целые числа n и k. Пусть есть список всех хороших строк длины n. Будем считать, что строки отсортированы в лексикографическом порядке. \\
 
 Необходимо вернуть k-ую строку этого списка. Если в списке менее k хороших строк, то вернуть None.\\
 
 \textbf{Входные данные}
 
 В строке находятся целые числа n и k.\\
 
  \textbf{Выходные данные}
  
 Хорошая строка под номером k.\\
 
  \textbf{Пример 1}\\
  \begin{table}[!h]
  \centering
  \begin{tabular}{ | l | l | }
\hline
Входные данные & Выходные данные \\ \hline
1 3 & 'о'  \\
\hline
\end{tabular}\\
\end{table}


\textbf{Пример 2}\\
\begin{table}[!h]
  \centering
  \begin{tabular}{ | l | l | }
\hline
Входные данные & Выходные данные \\ \hline
3 4 & 'dog'  \\
\hline
\end{tabular}\\
\end{table}

\textbf{Пример 3}
\begin{table}[!h]
  \centering
  \begin{tabular}{ | l | l | }
\hline
Входные данные & Выходные данные \\ \hline
1 4 & None  \\
\hline
\end{tabular}
  \end{table}

\newpage
\textbf{Задача 2}\\
 
 \textbf{Постановка}
 
 Пусть дана строка. Необходимо переставить в ней буквы так, чтобы одинаковые буквы не стояли рядом. Перестановку нужно сделать в лексикографическом порядке. Если это невозможно, то верните None.\\
 
 \textbf{Входные данные}
 
 Строка.\\
 
  \textbf{Выходные данные}
  
\begin{itemize}
    \item если перестановка существует, то измененная строка.
    \item если не существует, то None.
\end{itemize}
 
  \textbf{Пример 1}\\
  \begin{table}[!h]
  \centering
  \begin{tabular}{ | l | l | }
\hline
Входные данные & Выходные данные \\ \hline
'abbc' & 'abcb'  \\
\hline
\end{tabular}\\
\end{table}


\textbf{Пример 2}\\
\begin{table}[!h]
  \centering
  \begin{tabular}{ | l | l | }
\hline
Входные данные & Выходные данные \\ \hline
'bbbc' & None  \\
\hline
\end{tabular}\\
\end{table}

\newpage
\textbf{Задача 3}\\
 
 \textbf{Постановка}
 
Некоторая строка закодирована шифром. Чтобы расшифровать эту строку, существует некоторая машина, которая читает один символ за раз и выполняет следующие действия:
\begin{itemize}
    \item если прочитанный символ это буква, то она записывается на ленту
    \item если прочитанный символ это цифра n, то вся записанная строка печатается ещё n-1 раз.
\end{itemize}

Для некоторой зашифрованной строки необходимо вывести k-ую буквы расшифрованной строки.\\
 
 \textbf{Входные данные}
 
 Строка и целое число k.\\
 
  \textbf{Выходные данные}
  
k-ая буква расшифрованной строки.\\
 
  \textbf{Пример 1}\\
  \begin{table}[!h]
  \centering
  \begin{tabular}{ | l | l | }
\hline
Входные данные & Выходные данные \\ \hline
'cute2dog3' \quad 10 & 'o'  \\
\hline
\end{tabular}\\
\end{table}


\textbf{Пример 2}\\
\begin{table}[!h]
  \centering
  \begin{tabular}{ | l | l | }
\hline
Входные данные & Выходные данные \\ \hline
'no23' \quad 5 & 'n'  \\
\hline
\end{tabular}\\
\end{table}
 
\end{document}
